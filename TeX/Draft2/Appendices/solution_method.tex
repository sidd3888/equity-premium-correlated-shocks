\subsection{Solving the model}\label{app:seq_egm}

I solve the model using a sequential application of the endogenous grid method \citep{Carroll2006}, dividing a period into two subperiods, the first stage involving a consumption decision ($c$), and the second involving the portfolio optimization problem ($\vs$). This description largely pertains to the finite-horizon version, though the infinite-horizon solution merely replaces periods with a sequence of guesses.

Construct a grid of assets $\Ac = \bs{\underline{a} = a_1 < a_2 < ... < a_k = \overline{a}}$. To solve the problem pertaining to any period $t$, observe from equation (\ref{eq:excess_return}) that whenever $a_i \neq 0$, the optimal share of risky assets is given by the choice of $\hat{\vs}_{t+1}(a_i) \in [0,\,1]$ such that
\begin{equation}\label{eq:kappa}
n^{-3}\G_{t+1}^{-\rho}\sum_{j=1}^{n^3}(\Rfree\nu_i - \Rfix)(\psi_{j}c_{t+1}(m_{ij}))^{-\rho} = 0
\end{equation}
where
\[
m_{ij} = \frac{\Rfix + \hat{\vs}_{t+1}(a_i)(\Rfree\nu_j - \Rfix)}{\G_{t+1}\psi_{j}}a_i + \t_j
\]
The problem then becomes a root-finding operation pertaining to a function of $\hat{\vs}$, which, given a policy function $c_{t+1}$, yields an optimal level of $\hat{\vs}$ for each $a_i$. Denote this pair as $(a,\,\hat{\vs})_{i}$, and the resulting effective return $\Rfix + \hat{\vs}_{i}(\Rfree\nu_{j} - \Rfix)$ for each value of the shocks as $\Rc_{ij}$.

For each \textit{end-of-period} outcome $(a,\,\hat{\vs})_i$, given $c_{t+1}$, we can use the consumption Euler equation to get
\[
[\hat{c}_t(a_i,\,\hat{\vs}_i)]^{-\rho} = \b\G_{t+1}^{-\rho} n^{-3}\sum_{j=1}^{n^3}\Rc_{ij}(\psi_{j}c_{t+1}(m_{ij}))^{-\rho}
\]
where $\hat{c}$ denotes that this yields a consumed function of the assets and portfolio share.
This function is then given by
\[
\hat{c}_{t}(a_i,\,\hat{\vs}_i) = \bs{\b\G_{t+1}^{-\rho}n^{-3}\sum_{j=1}^{n^3}\Rc_{ij}(\psi_{j}c_{t+1}(m_{ij}))^{-\rho}}^{-\frac{1}{\rho}}
\]
Now we have a vector of $\hat{c}_i$ corresponding to each $(a,\,\hat{\vs})_i$. Since $m_{t} = c_{t} + a_{t+1}$, we can construct the grid $\Mc$ with each $m_i \in \Mc$ given by $m_i = \hat{c}_i + a_i$, where $\hat{c}_{i} = \hat{c}_{t}(a_i,\,\hat{\vs}_i)$. We can now rewrite $c_t(m_i) = \hat{c}_t(a_i,\,\hat{\vs}_i)$ and $\vs_{t+1}(m_i) = \hat{\vs}_{t+1}(a_i)$, and interpolate to get the policy functions $(c_{t}(m),\,\vs_{t+1}(m)) = g_{t}(m)$ for period $t$.\footnote{While I use linear interpolation by default, cubic-spline interpolation yields similar results for the consumption function. However, due to drastic directional changes in the optimal portfolio share, spline interpolation sometimes suggests a portfolio share outside the interval $[0,\,1]$.} In the finite-horizon case, the model can be solved using $c_{T}(m) = m$ as the initial policy function and iterating backwards till period 0. For the infinite-horizon case, I use a guess $c_{0}(m)$ to obtain a sequence of guesses $\bc{c_{k}(m),\,\vs_{k}(m)}_{k=0}^{K}$ that converge to the true policy functions $c(m)$ and $\vs(m)$. Since my focus is not on life-cycle applications, I solve each model with a constant permanent growth factor $\G$.