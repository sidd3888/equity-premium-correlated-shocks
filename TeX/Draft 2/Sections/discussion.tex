\section{Discussion}\label{discussion}

\subsection{Reconciling model predictions with data}\label{reconciliation}

Section \ref{us_data} shows that while correlations between permanent income growth shocks and equity return shocks do explain the preference for saving in the safe asset to some extent, the returns on equity in the data are too high at relatively low volatility. As such, even with a very high correlation between permanent income growth shocks and equity returns, we require unrealistically high levels of relative risk aversion to explain the portfolio allocation observed in the data. A particular problem area is that the equity share limit for wealthy individuals is extremely high, at close to 80 percent, even with $\rho = 7$. While portfolio shares dip to reasonable numbers with zero income events around the target wealth, the model cannot explain the portfolio choice decisions of consumers outside a small neighborhood of the target wealth in the distribution and non-participation in the stock market.

One approach used to address this problem is to incorporate non-expected utility preferences. \citet{Haliassos2001} examine how various models of decision-making under risk improve predictions on equity holdings. While they conclude that changing preferences alone is not sufficient to account for the equity premium, they show that \citet{Kreps1978} preferences and, to a greater extent, Rank-Dependent Utility \citep{Quiggin1982} provide more realistic predictions on portfolio composition. Similarly, \citet{Schreindorfer2020} shows that disappointment averse preferences \citep{Gul1991, Routledge2010} can help explain the equity premium at a much lower level of relative risk aversion than with expected utility preferences under their model. However, the risk aversion coefficient necessary with expected utility preferences in their model is 34, meaning that despite the marked improvement, the new risk aversion coefficient is as high as 10.

Another explanation for the lower portfolio share of equity in the data than predicted in the model is pessimism and heterogeneity in beliefs about stock returns. \citet{Haliassos1995} argue that in addition to correlations between labor income and asset returns and departures from expected utility, factors such as informational frictions provide a good explanation for the equity premium. While they note that a lack of knowledge about the stock market constrained participation, \citet{Dominitz2007} find that agents also hold exaggerated beliefs about the possibility of negative nominal stock returns. \citet{Mateo2024} incorporates estimated beliefs from survey data into a life-cycle model and finds that stock market participation and conditional equity portfolio share can be explained by heterogeneous beliefs with a high average belief about the volatility of returns to equity. For consumers who believe that stock returns are extremely volatile, a high, or even moderate correlation between permanent income shocks and asset returns should ensure that they do not participate in the stock market, whereas the conditional distribution over equity portfolio share would then be determined by those who believe the stock market is not as volatile, though possibly more than actually observed in the data.

As far as non-participation in the stock market is concerned, minimum investment limits and fixed costs for participation have also been studied as probable obstacles. In the current model, with the parameterization as in section \ref{us_data} stock market non-participation is observed at target wealth solely due to the no-borrowing constraint, and cannot be seen with the zero income event. However, a fixed participation cost would preclude consumers with very little wealth from investing their savings in equity, thereby generating a non-participation effect among low-wealth consumers. One limitation to this approach is that it cannot explain non-participation across wealth levels. \citet{Andersen2011,Briggs2021} show that consumers who experience windfall gains do not see significantly higher participation rates, and that some of them even liquidate inheritances received in the form of stock.\footnote{Note that individuals who experience windfall gains do not experience an increase in permanent labor income, which means that their normalized wealth must also increase. Thus, variance in wealth due to the variance in permanent income alone does not capture such individuals.}

\subsection{Distribution of portfolio share}

Given the form of the policy function, note that under either of the specifications, computing a wealth distribution of agents is tantamount to obtaining a distribution over the portfolio share of equity. To do so, we can simulate the trajectories of $N$ agents (assuming a suitably large $N$) over a substantial number of periods to approximate a stationary distribution of wealth and portfolio shares. The first thing we need to do prior to this exercise is to formalize the manner in which shocks are generated.

Firstly, returns on equity are common to all agents. The shocks to permanent income growth and transitory income, however, can be modelled as idiosyncratic. Then, a consumer's wealth transition is given by
\[
m_{i,\,t+1} = \frac{\Rfix + \vs(m_{i,\,t})(\Rfree\nu_{t+1} - \Rfix)}{\G\psi_{i,\,t+1}}(m_{i,\,t} - c(m_{i,\,t})) + \t_{i,\,t+1}
\]
Note that the shocks to income experienced by all the individuals are independent of each other only when the shocks to income are uncorrelated to the common shock, which is the return on equity. However, the shocks to permanent and transitory income can be modelled as conditionally independent upon the realization of the shock to the return on equity. Then, the transition of wealth in the economy can be simulated by generating an asset return shock for each period, and generating $N$ values for the permanent income growth shock from the conditional distribution of $\psi$ and $\z$, given the realization of $\nu$. In the NBC model, the generated values of $\z$ are exactly the values of $\t$. In the NIR model, however, we can independently generate $N$ draws of a Bernoulli random variable with probability $1 - \wp$ to determine which agents experience a zero-income event, thereby determining the value of $\t$. Given the realization of the shocks, the transition will be specified by $m_{i,\,t+1}$ as defined above.

For the general statement of the problem, my simulation algorithm proceeds as follows. I first generate a single draw from the marginal distribution of the return on equity. Then, given the realized value of $\ln(\nu)$, I calculate the conditional distribution of the bivariate normal variables $(\ln(\psi),\,\ln(\z))$ as in \citet[p.34]{Anderson2003}, and instantiate the derived bivariate lognormal. I then use the draws from this distribution and the Bernoulli draws to determine $\psi_{i,\,t+1}$ and $\t_{i,\,t+1}$, for each $i$. Finally, I calculate the transition of wealth for each agent, and repeat the process for a large number of periods ($T = 120$) to approximate the stationary distribution of wealth and \textit{incoming} portfolio shares. That is, the distribution over portfolio shares of equity stemming from previous period wealth, which reflects each agent's begeinning-of-period asset holdings. I look at the baseline parameterization used in Section \ref{results}.

\begin{figure}[h]
    \centering
    \begin{subfigure}{0.49\textwidth}
        \centering
        \includegraphics[width=0.8\textwidth]{\NBCwealthdist}
        \caption{NBC model}
        \label{subfig:NBCwealthdist}
    \end{subfigure}
    \begin{subfigure}{0.49\textwidth}
        \centering
        \includegraphics[width=0.8\textwidth]{\NIRwealthdist}
        \caption{NIR model}
        \label{subfig:NIRwealthdist}
    \end{subfigure}
    \caption{Stationary wealth distribution in the NBC and NIR models}
    \label{fig:wealthdist}
\end{figure}
Figure \ref{fig:wealthdist} shows the distribution of wealth-to-permanent income among individuals in the NBC and NIR models. As is already well understood, the stationary distribution of wealth is centered around the target level of wealth in both models, though these savings targets are different, due to the differences in how agents respond to borrowing constraints and zero-income events. Given this stationary distribution, we can define a stationary distribution over portfolio shares of equity, which is a function of the wealth-to-permanent income ratio for each individual. The stationary distribution of portfolio shares in shown in Figure \ref{fig:sharedist}. Both distributions are radically different. Due to the nature of the optimal portfolio share rule in the NBC model, there is a high rate of clustering around 0. That is, many agents do not participate in the stock market, so long as they do not face a large transitory income shock in the current period. In the NIR model, however, the distribution is more spread out, reflecting the smoothness of the portfolio share rule around the trget level of wealth.
\begin{figure}[h]
    \centering
    \begin{subfigure}{0.49\textwidth}
        \centering
        \includegraphics[width=0.8\textwidth]{\NBCsharedist}
        \caption{NBC model}
        \label{subfig:NBCsharedist}
    \end{subfigure}
    \begin{subfigure}{0.49\textwidth}
        \centering
        \includegraphics[width=0.8\textwidth]{\NIRsharedist}
        \caption{NIR model}
        \label{subfig:NIRsharedist}
    \end{subfigure}
    \caption{Stationary distribution of portfolio shares in the NBC and NIR models}
    \label{fig:sharedist}
\end{figure}
The joint distribution of wealth and equity portfolio share going into the next period is rather trivial, as it is concentrated along the optimal portfolio share rule, weighted according to the stationary distribution of wealth. On the other hand, there is merit in examining the joint distribution of current portfolio holdings (which are determined by the previous period's wealth), and current wealth. This distribution is shown in Figure \ref{fig:jointdist}. As predictable, this distribution exhibits the relationship between wealth and portfolio share according to the optimal portfolio share rule around target wealth, but also reflects the dispersion arising from idiosyncratic shocks in the next period. Furthermore, the distribution in the NIR model also reflects the small proportion of agents who are faced with zero-income events, who, despite their lower wealth, invest similar amounts in equity as those with higher wealth. How would these predictions change upon calibrating to U.S. data instead? Without increasing $\rho$, there would be a degenerate distribution of portfolio shares in the NBC model at 1, while any variation in the NIR model would be due to the zero-income events. Upon increasing $\rho$, the model qualitatively stays true to the baseline patterns, though the exact support of the distribution changes.
\begin{figure}[h]
    \centering
    \begin{subfigure}{0.49\textwidth}
        \centering
        \includegraphics[width=0.8\textwidth]{\NBCjointdist}
        \caption{NBC model}
        \label{subfig:NBCjointdist}
    \end{subfigure}
    \begin{subfigure}{0.49\textwidth}
        \centering
        \includegraphics[width=0.8\textwidth]{\NIRjointdist}
        \caption{NIR model}
        \label{subfig:NIRjointdist}
    \end{subfigure}
    \caption{Stationary joint distribution of wealth and portfolio shares in the NBC and NIR models}
    \label{fig:jointdist}
\end{figure}