\subsection{Proof of Proposition \ref{prop:low_wealth_share}}\label{pf:low_wealth_share}

Note from Appendix \ref{app:cons_return_cov}, that
\begin{align*}
    \rho\cov{\ln \D C_{t+1},\,\Rfree_{t+1}} &\approx \rho\Rfree\bp{\o_{\psi,\,\nu} + \frac{c'(\bar{m})}{c(\bar{m})}\bp{(m_t - c_t)\bp{\frac{\vs\Rfree\s_{\nu}^2 - (\Rfix + \vs(\Rfree - \Rfix))\o_{\psi,\,\nu}}{\G}} + \o_{\z,\,\nu}}}
\end{align*}
By Lemma \ref{lm:nbc_kink}, there exists $\tilde{m} > 0$ such that $c(m) = m$ for $m \leq \tilde{m}$. For such $m$,
\begin{align*}
    \rho\cov{\ln \D C_{t+1},\,\Rfree_{t+1}} &\approx \rho\Rfree\bp{\o_{\psi,\,\nu} + \frac{c'(1)}{c(1)}\o_{\z,\,\nu}}
\end{align*}
as $a_{t+1} = 0$ and $\bar{m} = \E\bs{\t_{t+1}} = 1$. Since $\rho\cov{\D\ln C_{t+1},\,\Rfree_{t+1}}$ is continuous in $m$,\footnote{\citet{Carroll2024b} show that $c$ is twice continuously differentiable in $m$.} if $\rho\cov{\ln \D C_{t+1},\,\Rfree_{t+1}} > \Rfree - \Rfix$ for $m < \tilde{m}$, it is also  true for $m < \tilde{m} + \e$, for $\e > 0$ and any $\vs \in \bs{0,\,1}$. Thus, under the conditions for optimality, $\vs(m) = 0$ for all such $m$. Likewise, $\vs(m) = 1$ for $m < \tilde{m} + \e$ if $\rho\cov{\ln \D C_{t+1},\,\Rfree_{t+1}} < \Rfree - \Rfix$ for $m' < \tilde{m}$. Defining $m^*$ as the supremum of such $m$, we have the binary policy rule, as in the result. The only thing left to prove is that $\tilde{\o} = \frac{\Rfree - \Rfix}{\rho\Rfree}$. That is obvious upon dividing both sides of the excess return equation by $\rho\Rfree$.

\subsection{Proof of Proposition \ref{prop:wealth_share_diff}}\label{pf:wealth_share_diff}

% Note from Appendix \ref{app:cons_return_cov}, we can rewrite the excess return equation as
% \begin{align*}
%     \Rfree - \Rfix &\approx \rho\Rfree\bp{\o_{\psi,\,\nu} + \frac{c'(\bar{m})}{c(\bar{m})}\bp{(m_t - c_t)\bp{\frac{\vs\Rfree\s_{\nu}^2 - (\Rfix + \vs(\Rfree - \Rfix))\o_{\psi,\,\nu}}{\G}} + \o_{\z,\,\nu}}}\\
%     \vs(m) &\approx \frac{\G}{\Rfree\s_{\nu}^2 - (\Rfree - \Rfix)\o_{\psi,\,\nu}}\bp{\frac{c(\bar{m})}{c'(\bar{m})\bp{m - c(m)}}\bp{\frac{\Rfree - \Rfix}{\rho\Rfree} - \o_{\psi,\,\nu}} + \frac{\o_{\z,\,\nu}}{m - c(m)} - \frac{\Rfix\o_{\psi,\,\nu}}{\G}}
% \end{align*}
% After the kink in the consumption function, the optimal portfolio share is differentiable in wealth, as the consumption function is twice-continuously differentiable in wealth. First, note
% \[
% \frac{d}{dm}\frac{c(\bar{m})}{c'(\bar{m})\bp{m - c(m)}} = \frac{c'(\bar{m})^2\b[]}
% \]


% The derivative of the optimal portfolio share is then
% \[
% \vs'(m) \approx \frac{\G}{\Rfree\s_{\nu}^2 - (\Rfree - \Rfix)\o_{\psi,\,\nu}}\bp{}
% \]


Note from Appendix \ref{app:cons_return_cov}, the excess return equation is approximated as
\begin{align*}
    \Rfree - \Rfix &\approx \rho\Rfree\bp{\o_{\psi,\,\nu} + \frac{c'(\bar{m})}{c(\bar{m})}\bp{(m_t - c_t)\bp{\frac{\vs\Rfree\s_{\nu}^2 - (\Rfix + \vs(\Rfree - \Rfix))\o_{\psi,\,\nu}}{\G}} + \o_{\z,\,\nu}}}
\end{align*}
Note,
\[
\bar{m}'(m) = \frac{\Rfix + \vs(\Rfree - \Rfix)}{\G} > 0
\]
and
\[
\frac{d}{dm}\frac{c'(\bar{m})(m - c(m))}{c(\bar{m})} = \frac{c(\bar{m})(c''(\bar{m})(m - c(m))\bar{m}'(m) + c'(\bar{m})(1 - c'(m))) - c'(\bar{m})^2(m - c(m))\bar{m}'(m)}{c(\bar{m})^2}
\]
By the concavity of $c$, the above derivative is negative. Supposing that $\o_{\z,\,\nu} = 0$,\footnote{This covariance becomes inconsequential with wealth, as $\frac{c'(\bar{m})}{c(\bar{m})}$ sharply falls} we only care about the positivity or negativity of the term
\[
    \vs\Rfree\s_{\nu}^2 - (\Rfix + \vs(\Rfree - \Rfix))\o_{\psi,\,\nu}
\]
This term is positive if and only if
\begin{align*}
    \o_{\psi,\,\nu} &< \frac{\vs\Rfree\s_{\nu}^2}{\Rfix + \vs(\Rfree - \Rfix)}\\
    &\leq \s_{\nu}^2
\end{align*}
Therefore, if $\o_{\psi,\,\nu} \geq \s_{\nu}^2$, this term is non-positive, and $\cov(\ln \D C_{t+1},\,\Rfree_{t+1})$ is increasing in wealth. As such, the optimal portfolio share is decreasing in wealth. On the other hand, if $\o_{\psi,\,\nu} < \s_{\nu}^2$, the term is positive when $\vs = 1$, and negative when $\vs = 0$. Thus, there exists a threshold $\vs$ such that the optimal portfolio share is increasing in wealth below this portfolio share. This threshold is given by
\[
\vs < \frac{\Rfix\o_{\psi,\,\nu}}{\Rfree\s_{\nu}^2 - (\Rfree - \Rfix)\o_{\psi,\,\nu}}
\]
Now, suppose $\o_{\psi,\,\nu} = \tilde{\o} = \frac{\Rfree - \Rfix}{\rho\Rfree}$. Then
\begin{align*}
    \frac{\Rfix\frac{\Rfree - \Rfix}{\rho\Rfree}}{\Rfree\s_{\nu}^2 - \frac{(\Rfree - \Rfix)^2}{\rho\Rfree}} &= \frac{\Rfix(\Rfree - \Rfix)}{\rho\Rfree^2\s_{\nu}^2 - (\Rfree - \Rfix)^2}
\end{align*}
Since the threshold value is increasing in $\o_{\psi,\,\nu}$, the optimal portfolio share is increasing in wealth for $\o_{\psi,\,\nu} > \tilde{\o}$ as long it is below this share share. Likewise, the optimal portfolio share is decreasing in wealth for $\o_{\psi,\,\nu} < \tilde{\o}$ as long as it is above this share. As such, we must only show that in each case, the optimal portfolio share lies on either side of this threshold.

Rearranging the excess return equation, the optimal portfolio rule can be written as
\[
    \vs(m) \approx \frac{\G}{\Rfree\s_{\nu}^2 - (\Rfree - \Rfix)\o_{\psi,\,\nu}}\bp{\frac{c(\bar{m})}{c'(\bar{m})\bp{m - c(m)}}\bp{\frac{\Rfree - \Rfix}{\rho\Rfree} - \o_{\psi,\,\nu}} - \frac{\Rfix\o_{\psi,\,\nu}}{\G}}
\]
First, note that differentiating the right hand side w.r.t $\o_{\psi,\,\nu}$ reveals that it is decreasing in $\o_{\psi,\,\nu}$. Thus, the optimal portfolio share is decreasing in $\o_{\psi,\,\nu}$. Now, note that the right hand side is decreasing in $\o_{\psi,\,\nu},$\footnote{Also see that $\cov{\ln C_{t+1},\,\Rfree_{t+1}}$ is increasing in $\o_{\psi,\,\nu}$, so $\vs$ is decreasing in $\o_{\psi,\,\nu}$.} so $\vs$ is decreasing in $\o_{\psi,\,\nu}$. Note, then, for $\o_{\psi,\,\nu} = \tilde{\o}$,
\begin{align*}
    \vs(m) &\approx \frac{\Rfix(\Rfree - \Rfix)}{\rho\Rfree^2\s_{\nu}^2 - (\Rfree - \Rfix)^2}
\end{align*}
Therefore, $\vs(m)$ is greater than the threshold share if $\o_{\psi,\,\nu} < \tilde{\o}$, and lower otherwise. As such, $\vs(m)$ is increasing in wealth for $\o_{\psi,\,\nu} > \tilde{\o}$, and decreasing otherwise. Note that for plausible values of $\Rfree$, $\Rfix$, and $\s_{\nu}$, $(\Rfree^2 - 1)\rho\s_{\nu}^2 - (\Rfree - \Rfix)^2 \approx 0$ and $\Rfix \approx 1$. Thus, the threshold value of $\vs$ is approximately
\begin{align*}
    \vs(m) &= \frac{\Rfix(\Rfree - \Rfix)}{\rho\Rfree^2\s_{\nu}^2 - (\Rfree - \Rfix)^2}\\
    &= \frac{\Rfix(\Rfree - \Rfix)}{\rho\s_{\nu}^2 + (\Rfree^2 - 1)\rho\s_{\nu}^2 - (\Rfree - \Rfix)^2}\\
    &\approx \frac{\Rfree - \Rfix}{\rho\s_{\nu}^2}
\end{align*}
Note that this is the Merton-Samuelson share.
