\begin{titlepage}
    \maketitle
    \begin{abstract}
        Portfolio choice models with buffer-stock saving normally predict that the portfolio share of equity declines with wealth, while the poor invest all their savings in equity. I examine an infinite-horizon consumption-savings problem where the agent has to decide how to invest their savings between a risky and risk-free asset. I find that with moderate levels of relative risk aversion, a correlation between permanent income growth and returns on equity lowers the optimal portfolio share of equity at target wealth levels, while inverting the relationship between wealth and equity portfolio share. I find that both the models with borrowing constraints and positive unemployment probability cannot fully explain the equity premium observed in U.S. data without elevated levels of risk aversion. However, the predictions are a definite improvement over the model without correlations between the shocks, and may explain the equity portfolio share in conjunction with other popular explanations in the literature.
    \end{abstract}
\end{titlepage}