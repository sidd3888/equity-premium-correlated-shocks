\begin{titlepage}
    \maketitle
    \vspace*{\stretch{0.5}}
    \begin{abstract}
        Portfolio choice models normally predict that the portfolio share of equity declines with wealth, while the poor invest all their savings in equity. I examine an infinite-horizon consumption-saving problem where the agent has to decide how to invest their savings between a risky and risk-free asset. Following \citet{Viceira2001}, I model permanent income growth shocks as correlated with returns on equity. I find that with moderate levels of relative risk aversion, this correlation lowers the optimal portfolio share of equity at target wealth levels (\citeauthor{Kimball1991}'s (\citeyear{Kimball1991}) temperance motive), while inverting the relationship between wealth and equity portfolio share. I find that, under standard calibrations, neither the model with borrowing constraints nor the one with positive unemployment probability can fully explain the equity premium observed in U.S. data without high levels of risk aversion. The calibrated temperance motive may observe the observed equity portfolio share in conjunction with other popular explanations documented in the literature.
    \end{abstract}
    \vspace*{\stretch{0.5}}
    Link to code: \href{https://github.com/sidd3888/equity-premium-correlated-shocks}{\texttt{https://github.com/sidd3888/equity-premium-correlated-shocks}}
    \vspace*{\stretch{1.5}}
\end{titlepage}