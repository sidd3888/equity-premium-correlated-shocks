\section{Discussion}\label{discussion}

\subsection{Reconciling model predictions with data}\label{reconciliation}

Section \ref{us_data} shows that while correlations between permanent income growth shocks and equity return shocks do explain the preference for saving in the safe asset to some extent, the returns on equity in the data are too high at relatively low volatility. As such, even with a very high correlation between permanent income growth shocks and equity returns, we require unrealistically high levels of relative risk aversion to explain the portfolio allocation observed in the data. A particular problem area is that the equity share limit for wealthy individuals is extremely high, at close to 80 percent, even with $\rho = 7$. While portfolio shares dip to reasonable numbers with zero income events around the target wealth, the model cannot explain the portfolio choice decisions of consumers outside a small neighborhood of the target wealth in the distribution and non-participation in the stock market.

One approach used to address this problem is to incorporate non-expected utility preferences. \citet{Haliassos2001} examine how various models of decision-making under risk improve predictions on equity holdings. While they conclude that changing preferences alone is not sufficient to account for the equity premium, they show that \citet{Kreps1978} preferences and, to a greater extent, Rank-Dependent Utility \citep{Quiggin1982} provide more realistic predictions on portfolio composition. Similarly, \citet{Schreindorfer2020} shows that disappointment averse preferences \citep{Gul1991, Routledge2010} can help explain the equity premium at a much lower level of relative risk aversion than with expected utility preferences under their model. However, the risk aversion coefficient necessary with expected utility preferences in their model is 34, meaning that despite the marked improvement, the new risk aversion coefficient is as high as 10.

Another explanation for the lower portfolio share of equity in the data than predicted in the model is pessimism and heterogeneity in beliefs about stock returns. \citet{Haliassos1995} argue that in addition to correlations between labor income and asset returns and departures from expected utility, factors such as informational frictions provide a good explanation for the equity premium. While they note that a lack of knowledge about the stock market constrained participation, \citet{Dominitz2007} find that agents also hold exaggerated beliefs about the possibility of negative nominal stock returns. \citet{Mateo2024} incorporates estimated beliefs from survey data into a life-cycle model and finds that stock market participation and conditional equity portfolio share can be explained by heterogeneous beliefs with a high average belief about the volatility of returns to equity. For consumers who believe that stock returns are extremely volatile, a high, or even moderate correlation between permanent income shocks and asset returns should ensure that they do not participate in the stock market, whereas the conditional distribution over equity portfolio share would then be determined by those who believe the stock market is not as volatile, though possibly more than actually observed in the data.

As far as non-participation in the stock market is concerned, minimum investment limits and fixed costs for participation have also been studied as probable obstacles. In the current model, with the parameterization as in section \ref{us_data} stock market non-participation is observed at target wealth solely due to the no-borrowing constraint, and cannot be seen with the zero income event. However, a fixed participation cost would preclude consumers with very little wealth from investing their savings in equity, thereby generating a non-participation effect among low-wealth consumers. One limitation to this approach is that it cannot explain non-participation across wealth levels. \citet{Andersen2011,Briggs2021} show that consumers who experience windfall gains do not see significantly higher participation rates, and that some of them even liquidate inheritances received in the form of stock.\footnote{Note that individuals who experience windfall gains do not experience an increase in permanent labor income, which means that their normalized wealth must also increase. Thus, variance in wealth due to the variance in permanent income alone does not capture such individuals.}

\subsection{Distribution of portfolio share}

Given the form of the policy function, note that under either of the specifications, computing a wealth distribution of agents is tantamount to obtaining a distribution over the portfolio share of equity. To do so, we can simulate the trajectories of $N$ agents (assuming a suitably large $N$) over a substantial number of periods to approximate a stationary distribution of wealth and portfolio shares. The first thing we need to do prior to this exercise is to formalize the manner in which shocks are generated.

Firstly, returns on equity are common to all agents. The shocks to permanent income growth and transitory income, however, can be modelled as idiosyncratic. Then, a consumer's wealth transition is given by
\[
m_{i,\,t+1} = \frac{\Rfix + \k(m_{i,\,t})(\Rfree_{t+1} - \Rfix)}{\Gc_{i,\,t+1}}(m_{i,\,t} - c(m_{i,\,t})) + \z_{i,\,t+1}
\]
While transitory income shocks are independent of all else, note that idiosyncratic shocks to permanent income growth are correlated with the aggregate shock to the return on equity. As such, while permanent income growth shocks are not independent across individuals, they can be modelled as conditionally independent upon the realization of the shock to the return on equity. Then, the transition of wealth in the economy can be simulated by generating an asset return shock for each period, and generating $N$ values for the permanent income growth shock from the conditional distribution of $\eta$, given the realization of $\nu$. Thereafter, we can independently generate $N$ values for $\z$ from the specified log-normal distribution. Given the realization of the shocks, the transition will be specified by $m_{i,\,t+1}$ as above.

\subsection{Extension to a life-cycle model}

A standard approach found in life-cycle modelling is to include time-varying sequences of permanent income growth factors $\bc{\G_{t}}_{t=\tau}^{T}$ and retirement ages after which income shocks are shut off. Additionally, there may be time-variation in shocks such as unforeseen health expenditures as the consumer ages and survival probabilities. The natural next step is to model transitory shocks to income and correlations between asset returns and permanent income growth shocks as time-variant. The consumption savings problem discussed in this paper can naturally be modified to
\[
\max_{\bc{C_{t}}_{t=\tau}^{T}}\E_{\tau}\sum_{t=\tau}^{T}\b^{t - \tau}\aleph_{\tau}^{t}u(C_{t})
\]
where $\aleph_{\tau}^{t}$ denotes the probability of surviving till period $t$ conditional on being alive in period $\tau$. The budget constraints remain unchanged, while the income process is now distributed according to
\[
\ln\bp{\eta_t,\,\nu_t,\,\z_t} \sim \Nc\bp{-\textbf{diag}(\S_t)/2, \S_t}
\]
For simplicity, we can assume that the volatility of logged permanent income growth shocks and asset return shocks are time-invariant, which implies that $\S_{t}$ is of the form
\[
\S_{t} = \begin{bmatrix}
    \s_{\eta}^2 & \o_t & 0\\
    \o_t & \s_{\nu}^2 & 0\\
    0 & 0 & \s_{\z,\,t}^2
\end{bmatrix}
\]
The only layer of added complexity in solving this model is a need to store a time-varying sequence of $\bc{\o_{t},\,\s_{\z,\,t}}$, which would mean that for every period, we discretize a new distribution over shocks when computing expectations of the Euler equations to solve for the optimal policy functions. The idea of a time-varying distribution of income shocks and asset returns can be found in a life-cycle setting starting with \citet{Constantinides2002}. While they directly influence the correlation between consumption and asset returns by introducing and removing wage income uncertainty in different periods of life, the decomposition of income shocks into permanent and transitory components in this model allows to maintain a fairly low contemporaneous correlation between wage income and asset returns, while also inducing a correlation between consumption and asset returns by affecting the present discounted value of human capital.